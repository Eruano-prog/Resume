%-------------------------
% Resume in Latex
% Author : Sourabh Bajaj + some brand new features from Mary Feofanova + some changes by Muhammadjon Hakimov
% Sourabh's: https://github.com/sb2nov/resume
% Mary's: https://github.com/mary3000/resume
% Muhammadjon's: https://github.com/mrhakimov/resume
% License : MIT
%------------------------

\documentclass[letterpaper,10pt]{article}

\usepackage{makecell}
\usepackage[link=off]{phonenumbers}
\usepackage{ragged2e}

\usepackage{latexsym}
\usepackage[empty]{fullpage}
\usepackage{titlesec}
\usepackage{marvosym}
\usepackage[usenames,dvipsnames]{color}
\usepackage{verbatim}
\usepackage{enumitem}
\usepackage[pdftex]{hyperref}
\usepackage{fancyhdr}
\usepackage[utf8]{inputenc} % Исправлено: utf8 вместо utf8x
\usepackage[english,russian]{babel}
\usepackage{cmap}

\pagestyle{fancy}
\fancyhf{} % clear all header and footer fields
\fancyfoot{}
\renewcommand{\headrulewidth}{0pt}
\renewcommand{\footrulewidth}{0pt}
\usepackage[margin=0.3in]{geometry}
% Adjust margins
\addtolength{\oddsidemargin}{-0.0in}
\addtolength{\evensidemargin}{-0.0in}
\addtolength{\textwidth}{0in}
\addtolength{\topmargin}{10pt}
\addtolength{\textheight}{0.0in}

\urlstyle{same}

\usepackage{xcolor} % http://ctan.org/pkg/xcolor
\usepackage{hyperref} % http://ctan.org/pkg/hyperref
\hypersetup{
  colorlinks=true,
  linkcolor=blue!50!red,
  linkbordercolor=red,
  urlcolor=black!70!black,
  pdfnewwindow=true
}

\raggedbottom
\raggedright
\setlength{\tabcolsep}{0in}

% Sections formatting
\titleformat{\section}{
  \vspace{-10pt}\scshape\raggedright\large
}{}{0em}{}[\color{black}\titlerule \vspace{-7pt}]

%-------------------------
% Custom commands
\def \ifempty#1{\def\temp{#1} \ifx\temp\empty }

\newcommand{\resumeItem}[2]{
  \item\small{
  	\ifempty{#1}#2\else\textbf{#1}{: #2 \vspace{-2pt}}\fi
  }
}

\usepackage[dvipsnames]{xcolor}
\definecolor{mygray}{gray}{0}
\usepackage{fancybox}

\usepackage{lmodern}
\usepackage{tikz}

% Style definition
\tikzset{rndblock/.style={rounded corners,rectangle,draw,outer sep=0pt}}

% Command Definition
% 1 optional to customize the aspect, 2 mandatory: text to be framed
\newcommand{\tframed}[2][]{\tikz[baseline=(h.base)]\node[rndblock,#1] (h) {\color{black}{#2}};}

\newcommand*{\mystrut}{\rule[-0.2\baselineskip]{0pt}{0.8\baselineskip}}
\newcommand{\skill}[1]{\tframed[lightgray]{\mystrut#1}}


\newcommand{\educationEntry}[5]{
  \item
    \begin{tabular*}{\textwidth}{l@{\extracolsep{\fill}}r}
      \textbf{#1} & \textcolor{mygray}{\textit{\small #5}} \\ % Название программы и даты
      \if\relax\detokenize{#2}\relax \else \textit{\small#2} \\ \fi % Название организации (если не пусто)
      \if\relax\detokenize{#3}\relax \else {\small#3} \\ \fi % Дополнительная информация (если не пусто)
      \if\relax\detokenize{#4}\relax \else \parbox[t]{0.8\textwidth}{\small#4} \\ \fi % Изученные темы (если не пусто)
    \end{tabular*}\vspace{-5pt}
    \vspace{12pt}  
}

\newcommand{\resumeExpSubheading}[5]{
  \vspace{3pt}
  \item
    \begin{tabular*}{0.97\textwidth}{l@{\extracolsep{\fill}}r}
      \vspace{2pt} \textbf{#1}  & \textcolor{mygray}{\small #2} \\
      \textit{#3} & \textcolor{mygray}{\textit{\small #4}} \\
      {\scriptsize#5}
    \end{tabular*}\vspace{3pt}
}

\newcommand{\resumeExpJointSubheading}[5]{
  \vspace{-14pt}
  \item
    \begin{tabular*}{0.97\textwidth}{l@{\extracolsep{\fill}}r}
      \vspace{2pt} \textbf{#1}  & \textcolor{mygray}{\small #2} \\
      \textit{#3} & \textcolor{mygray}{\textit{\small #4}} \\
      {\scriptsize#5}
    \end{tabular*}\vspace{3pt}
}

\newcommand{\resumeProjSubheading}[5]{
  \vspace{-10pt}\item
    \begin{tabular*}{0.97\textwidth}{l@{\extracolsep{\fill}}r}
      \vspace{2pt} \textbf{#1}  & \textcolor{mygray}{\small #2} \\
      \textbf{#3} & \textcolor{mygray}{\textit{\small #4}} \\
      {\scriptsize#5}
    \end{tabular*}\vspace{3pt}
}

\newcommand{\resumeSubItem}[2]{\resumeItem{#1}{#2}\vspace{-4pt}}

\renewcommand{\labelitemii}{$\circ$}

\newcommand{\resumeSubHeadingListStart}{
  \begin{itemize}[leftmargin=*, labelsep=5pt, itemindent=0pt, topsep=0pt, partopsep=0pt, align=left]
}
\newcommand{\resumeSubHeadingListEnd}{\end{itemize}}
\newcommand{\resumeItemListStart}{\begin{itemize}[leftmargin=0.2in]}
\newcommand{\resumeItemListEnd}{\end{itemize}\vspace{-5pt}}

\usepackage{changepage}
\newcommand{\resumeDesc}[1]{\begin{adjustwidth}{5pt}{0pt}\vspace{-2pt}{#1}\end{adjustwidth}}

\newcommand{\ExternalLink}{
    \tikz[x=1.2ex, y=1.2ex, baseline=-0.05ex]{
        \begin{scope}[x=1ex, y=1ex]
            \clip (-0.1,-0.1) 
                --++ (-0, 1.2) 
                --++ (0.6, 0) 
                --++ (0, -0.6) 
                --++ (0.6, 0) 
                --++ (0, -1);
            \path[draw, 
                line width = 0.5, 
                rounded corners=0.5] 
                (0,0) rectangle (1,1);
        \end{scope}
        \path[draw, line width = 0.5] (0.5, 0.5) 
            -- (1, 1);
        \path[draw, line width = 0.5] (0.6, 1) 
            -- (1, 1) -- (1, 0.6);
        }
    }
    
\definecolor{Blue1}{HTML}{4D4EDC}
\newcommand{\MYhref}[3][Blue1]{\href{#2}{\color{#1}{#3}}}

%-------------------------------------------
%%%%%%  CV STARTS HERE  %%%%%%%%%%%%%%%%%%%%%%%%%%%%


\begin{document}
%----------HEADING-----------------

\begin{center}\textbf{\Large Фёдор Дмитриевич Головлёв}\end{center}
\vspace{-12pt}
\begin{center}
Email: \MYhref{mailto:fdgolovlev@gmail.com}{fdgolovlev@gmail.com} \quad
Telegram: \MYhref{https://t.me/Erruano}{Erruano} \quad
GitHub: \MYhref{https://github.com/Eruano-prog}{Eruano-prog}
\end{center}

%-----------SUMMARY-----------------
\vspace{-10pt}
\section{О себе}
\justifying
Студент третьего курса. Ищу работу в компании, которая даст возможность получить опыт коммерческой разработки бэкенд приложений. В данный момент готов рассмотреть график с возможностью совмещать работу с учёбой.

%--------PROGRAMMING SKILLS------------
\section{Навыки}
\begin{tabular}{ll}
\textbf{Языки:} & \quad Go, Java, C\#, C++/C, Python, SQL, Bash \\
\textbf{Frameworks \& Tools:} & \quad PostgreSQL, gRPC, Docker, Kafka, MongoDB, Git \\
\end{tabular}

%-----------EDUCATION-----------------
\section{Образование}
  \resumeSubHeadingListStart
    \educationEntry
      {Разработка Программного Обеспечения/Software Engineering} % Название программы
      {Университет ИТМО} % Название организации
      {Факультет информационных технологий и программирования} % Дополнительная информация
      {}
      {3-й курс} % Даты или информация о курсе

    \educationEntry
      {Язык Go} % Название программы
      {Яндекс ШАД} % Название организации
      {} % Дополнительная информация (пусто)
      {Темы курса: Горутины и каналы, Продвинутое тестирование, Concurrency with shared memory, Работа с базами данных, Reflection, Go tooling} % Изученные темы
      {Осень 2024} % Даты или информация о курсе

    \educationEntry
      {Разработка микросервисных приложений на Go}
      {Yadro}
      {}
      {Темы курса: Rest, gRPC, распределение нагрузки, микросервисы, тестирование}
      {февраль - апрель 2025 года}
  \resumeSubHeadingListEnd
  
%-----------PROJECTS-----------------
\section{Проекты}
  \resumeSubHeadingListStart
      \resumeProjSubheading
      {}{}{Микросервисное приложение на Go}{}
      {\skill{Go} \skill{gRPC} \skill{Kafka} \skill{REST API} \skill{WebSocket} \skill{PostgreSQL}}
          \resumeDesc{
          \begin{itemize}
              \item Приложение представляет соббой API Gateway и набор микросервисов для интеграции с внешними сервисами.
              \item Вызов команд между микросервисами осуществляется через gRPC.
          \end{itemize}
          }

        \resumeProjSubheading
        {}{}{Магазин монет на Go}{}
        {\skill{Go} \skill{REST API} \skill{PostgreSQL} \skill{Docker} \skill{JWT} \skill{Unit testing} \skill{Integration testing}}
            \resumeDesc{
            \begin{itemize}
                \item Предоставляет основные операции по покупке вещей за монеты
                \item Обеспечивает целостность данных при RPS>1000 с средним временем ответа ~17ms.
                \item Настроен docker-compose
                \item Авторизация происходит через JWT токен
                \item Покрытие тестами основных частей >60\%. Также реализовано интеграционное тестирование
            \end{itemize}
            }

      \resumeProjSubheading
      {}{}{\href{https://github.com/Eruano-prog/Microservices-on-Java}{ Приложение с микросервисной архитектурой на Java\ExternalLink}}{}
      {\skill{Java} \skill{Spring Web} \skill{Spring Data} \skill{Spring Security} \skill{Kafka} \skill{REST API} \skill{Hibernate} \skill{PostgreSQL}}
          \resumeDesc{
          \begin{itemize}
              \item Приложение создано на \textbf{микросервисной} архитектуре, используя \textbf{Spring} и \textbf{Kafka} в качестве брокера сообщений.
              \item Общение с сервисом проходит по \textbf{REST API}.
          \end{itemize}
          }
          
      \resumeProjSubheading
      {}{}{\href{https://github.com/Eruano-prog/TaskSystem}{ Монолитное приложение на Java с JWT авторизацией\ExternalLink}}{}
      {\skill{Java} \skill{Spring Web} \skill{Spring Data} \skill{Spring Security} \skill{JWT} \skill{REST API} \skill{Hibernate} \skill{PostgreSQL}}
          \resumeDesc{
          \begin{itemize}
              \item Приложение аутентифицирует пользователя через \textbf{JWT токен}.
              \item Методы получения данных поддерживают фильтрацию и пагинацию.
              \item Приложение полностью задокументировано. Для этого используются \textbf{JavaDoc} и \textbf{Swagger}.
          \end{itemize}
          }
          
    \resumeProjSubheading
      {}{}{Основы ООП на C\#}{}
      {\skill{C\#}}
          \resumeDesc{
          \begin{itemize}
              \item Объектная модель симулятора фэнтезийного космического передвижения с модульными тестами.
              \item Конфигуратор ПК.
              \item Корпоративная система распределения сообщений.
              \item Приложение для взаимодействия и управления файловой системой.
              \item Система банкомата.
          \end{itemize}
          }
  \resumeSubHeadingListEnd
\end{document}